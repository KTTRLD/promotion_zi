% header.tex

\usepackage[a4paper,left=3.5cm,right=2.5cm,bottom=3cm,top=3cm]{geometry}
\usepackage[ngerman,english]{babel}
\usepackage{graphicx}
\usepackage{color}
\usepackage{import}
\usepackage{amsmath,amssymb}

\usepackage[numbers]{natbib}% von Author hinzugefügt
%\usepackage{amsmath,amssymb,subfigure}
 
\usepackage{verbatim} % von Autor ergänzt

\usepackage{needspace}% für itemize seiten neu

%%%%%%%%%%%%%%%%%%%%%%%%%%%%%%%%%%%%%%%%%%%%%%%%%%%%%%%%
\usepackage{wasysym} % für das checked symbol Häkchen
% für Farben im allgemeinen
\usepackage[table]{xcolor}
\usepackage{tabularx}
\usepackage{colortbl}	

\usepackage{array,longtable}
\usepackage[alwaysadjust]{paralist}
\setdefaultitem{\textbullet}{\textbullet}{\textbullet}{\textbullet}

%%%%%%%%%%%%%%%%%%%%%%%%%%%%%%%%%%%%%%%%%%%%%%%%%%%%%%

% Theorem-Umgebungen
\usepackage[amsmath,thmmarks]{ntheorem}
%\usepackage{svg}
% Korrekte Darstellung der Umlaute
\usepackage[utf8]{inputenc}
\usepackage[T1]{fontenc}
\usepackage{microtype}% verbesserter Randausgleich
\usepackage[autostyle=true,german=quotes]{csquotes}
% Algorithmen
\usepackage[plain,chapter]{algorithm}
\usepackage{algorithmic}
\usepackage{enumerate}
%Abkürzungsvereichnis
\usepackage[printonlyused]{acronym}

%PDF anhängen
\usepackage{pdfpages}

% Bibtex deutsch
\usepackage{bibgerm}
%Crossrodes referenzes
\usepackage{xr}
% URLs
\usepackage{url}
%notes und grafiken
\usepackage[ngerman]{todonotes}
\setlength{\marginparwidth}{2cm}
%Liste im Text
\usepackage{paralist}
% Caption Packet
\usepackage[margin=0pt,font=small,labelfont=bf]{caption}
\usepackage[labelformat=simple]{subcaption}
\renewcommand\thesubfigure{(\alph{subfigure})} 
%\usepackage[margin=0pt,font=small,labelfont=bf]{subcaption}
% Gliederung einstellen
%\setcounter{secnumdepth}{5}
%\setcounter{tocdepth}{5}

\usepackage[explicit]{titlesec}
\usepackage{lmodern}
\usepackage{lipsum}
\definecolor{mygreen}{rgb}{0,0.6,0}
\definecolor{mygray}{rgb}{0.5,0.5,0.5}
\definecolor{mymauve}{rgb}{0.58,0,0.82}
\definecolor{pl_background}{rgb}{0.95,0.95,0.95}
\definecolor{pl_comment}{rgb}{0.12, 0.38, 0.18 }
\definecolor{pl_ifelse}{rgb}{0.74,0.74,.29}
\definecolor{pl_keyword}{rgb}{0.37, 0.08, 0.25}
\definecolor{pl_string}{rgb}{0.06, 0.10, 0.98}

%\usepackage{listings}

\usepackage{listings,xcolor}
% Vordefiniertes Programmlisting
\lstdefinestyle{myCustomJavaStyle}{
%\lstset{
language = java,
basicstyle = \small\sffamily,
backgroundcolor = \color{pl_background},
stringstyle = \color{pl_string},
keywordstyle = \color{pl_keyword}\bfseries,
commentstyle = \color{pl_comment}\itshape,
frame = lrbt,
numbers = left,
showstringspaces = false,
breaklines = true,
xleftmargin = 15pt
%emph = [1]{java},
%emphstyle = [1]\color{black},
%emph = [2]{if,and,or,else},
%emphstyle = [2]\color{pl_keyword}
}

\lstdefinestyle{myCustomMatlabStyle}{
  language=java,
  numbers=left,
  stepnumber=1,
  numbersep=8pt,
  tabsize=2,
  showspaces=false,
  showstringspaces=true,
  backgroundcolor=\color{white},
  basicstyle=\footnotesize,
  commentstyle=\itshape\color{blue!90!black},
  keywordstyle=\bfseries\color{red!40!black},
  %identifierstyle=\color{black},
  %stringstyle=\color{blue}
  %keywordstyle=\color{black}
}

%\usepackage{rotating}%rotation der Tabelle Autor
%\usepackage{pdflscape} %Autor


% Abstand swischen zwei Absätzen
%\parskip=0.5em
%\setlength{\parindent}{0pt}


\newlength\chapnumb
\setlength\chapnumb{2.5cm}

\titleformat{\chapter}[block]
{\normalfont\sffamily}{}{0pt}
{\parbox[b]{\chapnumb}{%
   \fontsize{80}{70}\selectfont\thechapter}%
  \parbox[b]{\dimexpr\textwidth-\chapnumb\relax}{%
    \raggedleft%
    \hfill{\LARGE#1}\\
    \rule{\dimexpr\textwidth-\chapnumb\relax}{0.4pt}}}
\titleformat{name=\chapter,numberless}[block]
{\normalfont\sffamily}{}{0pt}
{\parbox[b]{\dimexpr\textwidth\relax}{%
    \raggedleft%
    \hfill{\LARGE#1}\\
    \rule{\dimexpr\textwidth\relax}{0.4pt}}}

% Theorem-Optionen %
\theoremseparator{.}
\theoremstyle{change}
\newtheorem{theorem}{Theorem}[section]
\newtheorem{satz}[theorem]{Satz}
\newtheorem{lemma}[theorem]{Lemma}
\newtheorem{korollar}[theorem]{Korollar}
\newtheorem{proposition}[theorem]{Proposition}
% Ohne Numerierung
\theoremstyle{nonumberplain}
\renewtheorem{theorem*}{Theorem}
\renewtheorem{satz*}{Satz}
\renewtheorem{lemma*}{Lemma}
\renewtheorem{korollar*}{Korollar}
\renewtheorem{proposition*}{Proposition}
% Definitionen mit \upshape
\theorembodyfont{\upshape}
\theoremstyle{change}
\newtheorem{definition}[theorem]{Definition}
\theoremstyle{nonumberplain}
\renewtheorem{definition*}{Definition}
% Kursive Schrift
\theoremheaderfont{\itshape}
\newtheorem{notation}{Notation}
\newtheorem{konvention}{Konvention}
\newtheorem{bezeichnung}{Bezeichnung}
\theoremsymbol{\ensuremath{\Box}}
\newtheorem{beweis}{Beweis}
\theoremsymbol{}
\theoremstyle{change}
\theoremheaderfont{\bfseries}
\newtheorem{bemerkung}[theorem]{Bemerkung}
\newtheorem{beobachtung}[theorem]{Beobachtung}
\newtheorem{beispiel}[theorem]{Beispiel}
\newtheorem{problem}{Problem}
\theoremstyle{nonumberplain}
\renewtheorem{bemerkung*}{Bemerkung}
\renewtheorem{beispiel*}{Beispiel}
\renewtheorem{problem*}{Problem}

% Algorithmen anpassen %
\renewcommand{\algorithmicrequire}{\textit{Eingabe:}}
\renewcommand{\algorithmicensure}{\textit{Ausgabe:}}
\floatname{algorithm}{Algorithmus}
\renewcommand{\listalgorithmname}{Algorithmenverzeichnis}
\renewcommand{\algorithmiccomment}[1]{\color{grau}{// #1}}

% Zeilenabstand einstellen %
\renewcommand{\baselinestretch}{1.25}

% Floating-Umgebungen anpassen %
\renewcommand{\topfraction}{0.9}
\renewcommand{\bottomfraction}{0.8}

% Markierte Referenzen werden ausgeblendet
\usepackage[pdfborder={0 0 0}]{hyperref}
% Abkuerzungen richtig formatieren %
\usepackage{xspace}

% Leere Seite ohne Seitennummer, naechste Seite rechts
\newcommand{\blankpage}{ \clearpage{\pagestyle{empty}\cleardoublepage}}
%\newcommand{\blankpage}{ \clearpage{\pagestyle{empty}\clearpage}}

% Keine einzelnen Zeilen beim Anfang eines Abschnitts (Schusterjungen)
\clubpenalty = 10000

% Keine einzelnen Zeilen am Ende eines Abschnitts (Hurenkinder)
\widowpenalty = 10000 
\displaywidowpenalty = 10000

% EOF

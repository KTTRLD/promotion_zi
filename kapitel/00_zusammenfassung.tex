% Bitte die folgende Zeile stehen lassen
\newgeometry{left=5cm,right=2.5cm,bottom=4cm,top=4cm}
\chapter*{Zusammenfassung}
\vspace*{-2.5em}
Alkohol ist als psychotrope Substanz mit potentiell abhängigkeitsfördernden Eigenschaften in dem westlichen Kulturkreis ubiquitär erhältlich und aufgrund der kulturellen und sozialen Akzeptanz für viele Altersgruppen leicht zugänglich. Es ist bereits bekannt, dass die Entwicklung eines auffälligen Konsumverhaltens oder gar die Entwicklung einer Alkoholsuchterkrankung auf komplexe Art und Weise zum einen von individuellen biologischen Faktoren, zum anderen auch von allgemeinen Umweltfaktoren abhängt. Groß angelegte Längsschnittstudien wie z.B. die europäische IMAGEN-Studie ermöglichen die Untersuchung dieser unterschiedlichen Einflussfaktoren im zeitlichen Verlauf auf der Ebene eines Individuums und stellen somit einen überaus wertvollen Datensatz dar. Die IMAGEN-Studie, die seit mittlerweile 10 Jahren in regelmäßigen Abständen Personen und deren Entwicklung ab dem 14. Lebensjahr untersucht, konnte beispielsweise zeigen, dass u.a. Suchtmittelkonsum in der Schwangerschaft, Traumatisierung sowie negative soziale Erfahrungen in der individuellen Entwicklung, beeinflussende Faktoren für die Entstehung von Suchterkrankungen oder andern psychischen Auffälligkeiten darstellen.
Die Konzeption der vorliegenden Arbeit greift den IMAGEN Datensatz auf und integriert vier weitere große Längsschnittstudien, darunter die Mannheimer Risikokinderstudie (MARS), die Rostocker Längsschnittstudie (ROLS), die FRANCES Datenerhebung, sowie die POSEIDON - Studie (Pre-, Peri-and POstnatal Stress in human off-spring: an approach to study Epigenetic Impact on DepressiON) in einen gemeinsamen Datensatz. Hierdurch soll ein erweiterter Datensatz entstehen, der als Basis für weitere Investigationen dienen kann. Die Harmonisierung erfordert die Entwicklung einer gemeinsamen Datenstruktur, wodurch erst die Nutzung eines erweiterten Patientendatensatzes möglich wird. Dieser Datensatz soll in der Folge in der hier geplanten Arbeit mittels etablierter maschineller Lernverfahren (z.B. Random Forrest, Gradient Boost oder Support-vector machine) neue Einflussfaktoren im Zusammenhang mit der Entwicklung von Alkoholsuchtverhalten liefern, die über den Ergebnisbereich der jeweiligen einzelnen Studien hinausgeht und somit einen Mehrwert liefern kann.
Die Zielsetzung der geplanten Arbeit ist daher:
1.	Harmonisierung mehrerer longitudinaler Studien (IMAGEN, FRANCES, MARS, POSEIDON, ROLS) zu einer gemeinsamen Datenbasis
2.	Bestätigung und Erweiterung etablierter Einflussfaktoren für die Entwicklung von Suchtverhalten, speziell von Alkohlsuchtverhalten.
3.	Identifikation von Protektivitäts- und Vulnerabilitätsfaktoren für die Entwicklung von Alkoholsuchtverhalten mithilfe von maschinellen Lernverfahren anhand der neuen Datenbasis.
Es handelt sich nach dem oben Beschriebenen, um eine retrospektive Untersuchung bereits vorhandener Datensätze die durch eine längerfristige Beobachtung gewonnen wurden. Die Konzeption dieser Arbeit wird im Rahmen der IMAC-mind(Improving Mental Health and Reducing Addiction in Childhood and Adolescence through Mindfulness: Mechanisms, Prevention and Treatment)-Studie realisiert. IMAC-Mind als Verbundsprojekt soll zur Verbesserung der psychischen Gesundheit und Verringerung von Suchtgefahr im Kindes- und Jugendalter durch achtsamkeitsbasierte Methoden beitragen.



%Funktionseinschränkung beider Arme und beider Beine betrafen in Deutschland Ende 2015 laut des Statistischen Bundesamt insgesamt 89088 Personen \cite{STAT2015}.
%Im Rahmen der vorliegenden Abschlussarbeit konnte eine Steuerung für einen Softwareprototyp realisiert werden, die es ermöglicht, mithilfe definierter Augengesten (Blickgeste, Fixation, Lidschluss, vertikale Augenbewegung) einen vom Benutzer entfernten beweglichen Roboter zu steuern. Hierbei ermöglicht eine am Roboter angebrachte Kamera eine \enquote{Live-Ansicht} der Umgebung aus \enquote{Sicht} des Roboters, die dem Benutzer auf einem Bildschirm präsentiert wird. Der mobile Roboter fungiert somit als \acf{tps} und erweitert den Blickbereich des Benutzers auf den des \acs{tps}. Aufseiten des Benutzers entsteht das Gefühl, sich in dieser entfernten Umgebung \textit{präsent} zu fühlen, obwohl er dies nicht ist. 

%Die vom System bereitgestellte Augengestenerkennung erfolgt mittels eines videobasierten stationären Eyetracking-Systems, das die aktuelle Blickposition des Benutzers, bezeichnet als \acf{por}, berechnet. Mithilfe des berechneten \acs{por} konnte durch fünf freiwillige Nutzer in einem der Modi des Prototyps die aktive Steuerung des mobilen Roboters durch einen Parcours erfolgen. Anschließend wurden die Nutzer gebeten, einen Fragebogen auszufüllen, anhand dessen zwei unterschiedliche Steuermodelle und ihre Vor- \bzw Nachteile in Bezug auf die Handhabung, die Machbarkeit und die Präzision einer derartigen Mensch-Roboter-Interaktion. Ferner konnte mittels des Fragebogens abgeschätzt werden, wie ermüdend die beiden Arten der Steuerung sind. Außerdem wurde eine Art \enquote{Panikschalter}, der einen sofortigen Stopp des Roboters ermöglicht, umgesetzt und mittels Fragebogen evaluiert.
 
%Die Hauptanwendergruppe des Prototyps liegt hierbei auf Personen mit Sprach- und Bewegungseinschränkungen, beispielsweise durch Erkrankungen des Rückenmarks nach traumatischen Ereignissen, wobei eine aktive Teilhabe am Alltag krankheitsbedingt fast nur durch technische Unterstützung möglich ist. In einer Vorgängerarbeit von Eidam et al. \vgl~\cite{Eidam2015,Eidam2016} konnte bereits eine alternative Kommunikationsmöglichkeit für Personen mit Sprach- und Bewegungseinschränkungen demonstriert werden. Darauf aufbauend, ermöglicht die vorliegende Arbeit die aktive Steuerung eines mobilen Robotersystems und demonstriert die Machbarkeit einer derartigen Steuerung. Dies stellt eine erleichterte und erweiterte Kommunikations- und Interaktionsform für diese Benutzergruppe dar. 

%Künftige Arbeiten könnten eine automatisierte Objektklassifikation in den Videobildern des \acs{tps} umsetzen, um die Nutzbarkeit des Systems zu verbessern.  

% Bitte ddie folgende Zeile stehen lassen
\restoregeometry

% Bitte die folgende Zeile stehen lassen
\newgeometry{left=5cm,right=2.5cm,bottom=4cm,top=4cm}
\chapter*{Zusammenfassung}
\vspace*{-2.5em}
%Funktionseinschränkung beider Arme und beider Beine betrafen in Deutschland Ende 2015 laut des Statistischen Bundesamt insgesamt 89088 Personen \cite{STAT2015}.
Im Rahmen der vorliegenden Abschlussarbeit konnte eine Steuerung für einen Softwareprototyp realisiert werden, die es ermöglicht, mithilfe definierter Augengesten (Blickgeste, Fixation, Lidschluss, vertikale Augenbewegung) einen vom Benutzer entfernten beweglichen Roboter zu steuern. Hierbei ermöglicht eine am Roboter angebrachte Kamera eine \enquote{Live-Ansicht} der Umgebung aus \enquote{Sicht} des Roboters, die dem Benutzer auf einem Bildschirm präsentiert wird. Der mobile Roboter fungiert somit als \acf{tps} und erweitert den Blickbereich des Benutzers auf den des \acs{tps}. Aufseiten des Benutzers entsteht das Gefühl, sich in dieser entfernten Umgebung \textit{präsent} zu fühlen, obwohl er dies nicht ist. 

Die vom System bereitgestellte Augengestenerkennung erfolgt mittels eines videobasierten stationären Eyetracking-Systems, das die aktuelle Blickposition des Benutzers, bezeichnet als \acf{por}, berechnet. Mithilfe des berechneten \acs{por} konnte durch fünf freiwillige Nutzer in einem der Modi des Prototyps die aktive Steuerung des mobilen Roboters durch einen Parcours erfolgen. Anschließend wurden die Nutzer gebeten, einen Fragebogen auszufüllen, anhand dessen zwei unterschiedliche Steuermodelle und ihre Vor- \bzw Nachteile in Bezug auf die Handhabung, die Machbarkeit und die Präzision einer derartigen Mensch-Roboter-Interaktion. Ferner konnte mittels des Fragebogens abgeschätzt werden, wie ermüdend die beiden Arten der Steuerung sind. Außerdem wurde eine Art \enquote{Panikschalter}, der einen sofortigen Stopp des Roboters ermöglicht, umgesetzt und mittels Fragebogen evaluiert.
 
Die Hauptanwendergruppe des Prototyps liegt hierbei auf Personen mit Sprach- und Bewegungseinschränkungen, beispielsweise durch Erkrankungen des Rückenmarks nach traumatischen Ereignissen, wobei eine aktive Teilhabe am Alltag krankheitsbedingt fast nur durch technische Unterstützung möglich ist. In einer Vorgängerarbeit von Eidam et al. \vgl~\cite{Eidam2015,Eidam2016} konnte bereits eine alternative Kommunikationsmöglichkeit für Personen mit Sprach- und Bewegungseinschränkungen demonstriert werden. Darauf aufbauend, ermöglicht die vorliegende Arbeit die aktive Steuerung eines mobilen Robotersystems und demonstriert die Machbarkeit einer derartigen Steuerung. Dies stellt eine erleichterte und erweiterte Kommunikations- und Interaktionsform für diese Benutzergruppe dar. 

Künftige Arbeiten könnten eine automatisierte Objektklassifikation in den Videobildern des \acs{tps} umsetzen, um die Nutzbarkeit des Systems zu verbessern.  

% Bitte ddie folgende Zeile stehen lassen
\restoregeometry

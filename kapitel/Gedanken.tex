Marvin Minskey 1980 TVision der Teleprresence
%%%%%%%%%%%%%%%%%%%%%%%%%%%%%%%%%%%%%%%%%%%%%%%%%%%%
%%%%%%%%%%%%%%% Zusammenfassung %%%%%%%%%%%%%%%%%%%%
%%%%%%%%%%%%%%%%%%%%%%%%%%%%%%%%%%%%%%%%%%%%%%%%%%%%

%Im Rahmen des Prototyps sollen zudem in Zukunft ausgewählte Interaktionsmöglichkeiten durch eine automatische Objektklassifikation in den Videobildern des TPS umgesetzt werden (\vgl \cite{Eidam2016}). Letzteres ist jedoch nicht Gegenstand dieser Arbeit.

%%%%%%%%%%%%%%%%%%%%%%%%%%%%%%%%%%%%%%%%%%%%%%%%%%%%
%%%%%%%%%%%%%%%%%%%% Einleitung %%%%%%%%%%%%%%%%%%%%
%%%%%%%%%%%%%%%%%%%%%%%%%%%%%%%%%%%%%%%%%%%%%%%%%%%%
Gefühle und Interaktion sind Grundbedürfnisse eines Menschen
ALS als Beispiel für eine Erkrankung mit motorischen Beeinträchtigungen führt zu einer Bewegungseinschränkung.
Technische Systeme helfen den Betroffenen an der Teilnahme am selbstbestimmten Leben. Ein Beispiel ist Stephen Hawking der trotz der körperlichen Limitation weiter. Früher konnte nur mittels statischer Inhalte beispielsweise Buchstabentafeln kommuniziert werden. 

%Traditionell werden Befehle oder Anweisungen und damit die eigentliche Kommunikation, von einem Menschen zu einem Computer, mittels einer Computermaus, einer Tastatur oder eines Joystick vermittelt. Für Menschen mit körperlichen Einschränkungen, speziell bei motorischen Beeinträchtigungen, können diese Methoden in Bezug auf die Nutzung einige Probleme bereiten. 
%Zusätzliche Methoden sind notwendig, um dieses Problem zu lösen und damit diesen Menschen einen Zugang zu technischen Anwendungen zu ermöglichen. In den letzten Jahren konnte, beispielsweise die Eyetracking-Systeme, als neue Methode weiterentwickelt werden und Menschen mit Sehstörungen einen effizienteren Zugang zu technischen Systemen ermöglichen.
%Gal Sont is a programmer who suffers from motor neurone disease.
%Ein prominentes Beispiel stellt die Firma Apple mit ihrer Spracherkennung namens SIRI zur Verfügung, die auf eine verbale Kommunikation, wie von Menschen praktiziert, ausgerichtet ist. Aber nicht nur die Sprache und die o.g. Geräte ermöglichen die Kommunikation mit einem Computer. 

%Das Austauschen und das Teilen von Gefühlen, innersten Absichten und Gedanken mit anderen Personen erscheint für viele Menschen selbstverständlich. Auch die frei Wahl des eigenen Aufenthaltsortes und die Interaktion in und mit dieser Umgebung ist ein Grundbedürfnis eines jeden Menschen. Doch für Personen mit Sprach- oder Bewegungseinschränkungen stellt sowohl die Kommunikation, als auch die aktive Interaktion mit der unmittelbaren Umgebung eine Herausforderung dar. Dies bedeutet nicht selten einen Einschnitt in die Lebensführung und die allgemeine Lebensqualität der betroffenen Personen. Unterstützende technische Systeme können hierbei einen Beitrag zur Verbesserung der Situation von Personen mit Sprach- oder Bewegungseinschränkungen leisten. Zur Gruppe der unterstützenden technischen Systeme zählen im Bereich der Medizin und Gesundheitsversorgung immer häufiger auch Eyetracking-Systeme und mobile Robotersysteme. Die Kombination dieser beiden Systeme bietet das Potential, die Handlungs- und Interaktionsfähigkeiten mit der näheren oder entfernteren Umgebung, speziell für Personen mit Sprach- oder Bewegungseinschränkung, zu erweitern um dieser Personengruppe zu einer aktiven und selbstbestimmten Lebensführung zu verhelfen. 

Eine Personengruppe, die im Sommer 2014 besondere mediale Aufmerksamkeit durch die sogenannte \enquote{Ice Bucket Challenge} (deutsch Eiskübelherausforderung) \footnote{\url{https://de.wikipedia.org/wiki/ALS_Ice_Bucket_Challenge}} erhalten hatte, sind Personen mit \acf{als}. Ziel dieser Spendenkampagne war es, die Forschung \bzgl der Ätiologie der Erkrankung und eventuellen Therapiemöglichkeiten voranzutreiben. \acs{als} ist eine schwere neuromuskuläre Erkrankung, die das Zentralnervensystem schädigt und in aller Regel zu einem vorzeitigen Tod führt. Alleine in Deutschland starben im Jahr 2015, 2044 Personen (je 100.000 Einwohner) an dieser Erkrankung, wobei die Zahl seit 1998 ansteigend ist, \vgl \acl{abb} \ref{fig:stat}.

\begin{figure}[ht]
   \begin{minipage}[t]{\linewidth} 
      \centering 
     %\includegraphics[scale=1]{bilder/grundlagen/als.pdf}
      \includegraphics[width=1\textwidth]{bilder/grundlagen/als.pdf}
   \end{minipage}% 
   \caption{Sterbefälle 1998 bis 2015 für die spinale Muskelatrophie und verwandte Syndrome (ICD-10 Code: G12), Statistisches Bundesamt (Destatis), 2017. In www.destatis.de (Thematische Recherche: Zahlen \& Fakten - Gesellschaft \& Staat - Gesundheit - Todesursachen - Dokumentart: Tabelle). Abrufdatum: 19. Februar 2017 \protect\footnotemark }\label{fig:stat} 
\end{figure} 

Die Erkrankung führt früher oder später in ein Stadium des \acf{lis}. Dies ist ein Zustand, der mit erhaltener kognitiver Fähigkeit einhergeht, bei gleichzeitig totaler oder subtotaler Bewegungseinschränkung. Diese Personengruppe ist auf unterstützende technische Systeme zur Kommunikation und Bewegung angewiesen.
Ein prominentes Beispiel, wie technische Systeme die Kommunikation mit der Umgebung ermöglichen können, zeigt der an \acs{als} erkrankte Prof. Stephen Hawking. Dieser verwendet zur Kommunikation eine speziell von Intel entwickelte Software mit dem Namen Assistive Context-Aware Toolkit (ACAT) als Kommunikationsschnittstelle zum Computer \footnote{\url{https://01.org/acat}, (Stand: 14.12.2016)} 
\footnote{\url{http://www.hawking.org.uk/the-computer.html} (Stand: 14.12.2016)}. ACAT ermöglicht durch die Darstellung einer Tastatur auf einem Bildschirm, die Auswahl von Buchstaben, welche eine Wortvorhersage ermöglicht und eine darauf aufbauende Sprachausgabe erzeugt. Diese technisch fortgeschrittene Kommunikationsschnittstelle entwickelte sich initial aus einfachen statischen Buchstabentafeln, welche durch eine betreuende Person der motorisch beeinträchtigten Person präsentiert werden mussten um schlussendlich Wörter und Sätze zu formulieren. Eine alternative Lösung im Kontext der Kommunikation zeigt das  von Eidam et. al. verwendete Prinzip, \vgl \cite{Eidam2016}. Hierbei wurden mittels eines Softwareprototypen Alltagsgegenstände in einer statischen Bildschirmszene dargestellt und passende Interaktionen durch Augengesten, die von einem Eyetracker erfasst wurden, ausgewählt. Hierdurch konnte gezeigt werden, dass eine Interaktion mit Alltagsgegenständen und mittels eines Eyetrackers eine Erweiterung der Handlungsmöglichkeiten bieten kann. Neben der Kommunikationseinschränkung stellt die freie Wahl des Aufenthaltsortes ein weiteres Grundbedürfnis eines jeden Menschen dar. 


Diese vorliegende Arbeit erweitert diesen Ansatz um ein mobiles Robotersystem, welches durch Augensteuerung die Interaktionsmöglichkeiten einer Person mit Sprach- und Bewegungseinschränkungen verbessern soll. Dieser Ansatz führt zu folgender Zielsetzung für die vorliegende Arbeit. 

 \acs{als} ist jedoch nur eine von vielen Ursachen, die zu diesem Zustand führen kann. Personen mit Bewegungseinschränkungen bleiben oftmals nur die Augen als einziger Kommunikationskanal zur Außenwelt erhalten. Eine Kommunikation nur auf der Grundlage von Augenbewegungen zu ermöglichen, stellt jedoch eine große Herausforderung dar. Technische Systeme, wie Eyetracking-Systeme können Augengesten ohne invasive Methoden sichtbar und interpretierbar machen. Diese Systeme erleichtern dadurch die Kommunikation mit der unmittelbaren Umgebung. Ein Schritt hin zu mehr Selbständigkeit kann in einer Kombination eines Eyetraking-Systems mit einem mobilen Robotersystem liegen. 
Mobile Robotersysteme können eine Erweiterung der körperlichen Beeinträchtigungen von Personen mit Sprach- und Bewegungseinschränkung bereitstellen. Hier gibt es in den letzten Jahren zahlreiche kommerzielle Systeme.  Diese Systeme werden auch als \acf{tps} bezeichnet. 

Aus einer Verbindung zwischen Eyetracking-System und \acs{tps} können neue Möglichkeiten für Person mit Sprach- und Bewegungseinschränkungen entstehen, ihre Wünsche und Pläne eigenständig umzusetzen. 

\footnote{Quelle: Sterbefälle 1998 bis 2015 für die spinale Muskelatrophie und verwandte Syndrome (ICD-Code: G12 nach ICD-10), Statistisches Bundesamt (Destatis), 2017. In www.destatis.de (Thematische Recherche: Zahlen \& Fakten - Gesellschaft \& Staat - Gesundheit - Todesursachen - Dokumentart: Tabelle). Abrufdatum:19. Februar 2017 \url{http://www.doublerobotics.com/}(Stand: Dezember 2016)}.
\footnotetext{\url{https://www.destatis.de/DE/ZahlenFakten/GesellschaftStaat/Gesundheit/Todesursachen/Tabellen/GestorbeneAnzahl.html}(letzter Aufruf: 19. Februar 2017)}
% Einen Roboter nur mit den Augen in einer Umgebung zu steuern, kann eine  Aufgabe sein. Eventuell ist diese Umgebung sogar unbekannt oder nicht vertraut. Besonders Menschen mit Bewegungseinschränkungen können von einer Strategie der Steuerung von mobilen Robotersystemen nur mit den Augen profitieren. Die vorliegende Arbeit bietet zwei mögliche Strategien \bzw Steuerformen an, die es ermöglichen sollen, einen mobilen Roboter mithilfe eines Eyetracking-Systems zu steuern.

%Aufgrund der inhaltlichen Nähe der Grundlagen dieser Arbeit mit der Vorgängerarbeit von Frau Eidam, werde ich mich bzgl. der Grundlagen stark inhaltlich an dieser Vorgängerarbeit orientieren. Hierbei sollen zu Beginn ebenfalls die grundsätzlichen Funktionsweisen des menschlichen Auges und deren Bewegungen aufgezeigt werden. Es sollen die Möglichkeiten der Aufzeichnung von Augenbewegung mittels Eyetracker aufgezeigt werden. Es folgt eine Darstellung der verschiedenen möglichen Methoden mit Spezifizierung der in der Arbeit verwendeten Tracking Methode, sowie ein Überblick über die Anwendungsgebiete von Eyetrackingsystemen mit der Hinführung zu Medizinischen Anwendungsgebieten und hier spezielle Erkrankungen, welche für Eyetrackingsysteme und zur Erweiterung der Möglichkeiten einzelner beeinträchtigter Personen eingesetzt werden können. Abschließend soll der Einsatz von Telepräsenzrobotern in diesem Anwendungsbereich beschrieben und der aktuelle Stand dieser Technik aufgezeigt werden.Anschließend folgt die Beschreibung, der zur Realisierung der Software notwendigen Modellbildung, welche sich ebenfalls an der Struktur des MVC orientiert und dies um die Bereiche des Roomba-Client und die MJPG- Video-Stream erweitern soll. Es folgt die Beschreibung der Augengestensteuerung und die Beschreibung der Umsetzung der Steuerungsmethoden. Angedacht sind eine Basissteuerung und eine Art \enquote{Joystick}-Steuerung. Folgend wird auf die Benutzeroberfläche und die Softwarearchitektur, sowie die technische Umsetzung eingegangen werden. 
%5Im Schlussteil der Arbeit soll die Evaluation der Usibility im Bereich: Präzision, Ermüdung, \enquote{Panikschalter} im Fokus stehen. Dabei könnte überlegt werden, ob eine Art \enquote{Parcour} mit einzeln \enquote{Punkten}, die von einem Startpunkt angefahren werden sollen in Form der Steuerungssignale oder der absoluten Zeit quantifiziert werden können. Ferner kann überlegt werden, ob die Steuerung mit den Augen gegen eine “normale” Nutzung mit Maus oder Tastatur bestehen kann und in wieweit sich dies unterscheidet.

%%%%%%%%%%%%%%%%%%%%%%%%%%%%%%%%%%%%%%%%%%%%%%%%%%%%
%%%%%%%%%%%%%%%%%%%% Grundlagen %%%%%%%%%%%%%%%%%%%%
%%%%%%%%%%%%%%%%%%%%%%%%%%%%%%%%%%%%%%%%%%%%%%%%%%%%
%Dennoch galt es wie bereits in der Vorgängerversion einige Aspekte bei der Gestaltung der grafischen Benutzeroberfläche im Hinblick auf eine Nutzung per Augengesten zu beachten. Hierbei wurden auf einige wenige Aspekte bekannter Heuristiken der Fokus gelegt. \todo{kann diskutiert werden}

%Wie im Kapitel \ref{chapter:fragestellung} beschrieben liegt der Fokus der vorliegenden Arbeit auf der Realisierung einer Steuerung eines \acs{tps} mithilfe eines Eyetracking-Systems. Die Gebrauchstauglichkeit (engl. \textit{usability}) des Systems für Personen mit Sprach und Bewegungseinschränkungen zu verbessern ist ein langfristige Ziel in der Konzeption des Prototypen. Die Erstellung einer effizienten grafischen Benutzeroberfläche für definierte Augengesten erscheint im Hinblick auf eine geplante Nutzung des Systems durch Personen mit Sprach- und Bewegungseinschränkung sinnvoll. Im vorliegenden Prototypen wurden die Aspekte der optischen Eigenschaften nicht explizit untersucht.


%%%%%%%%%%%%%%%%%%%%%%%%%%%%%
%\aclp{tps} sind dadurch charakterisiert, dass die Steuerung durch einen entfernten Benutzer erfolgen kann. Durch die Telepräsenztechnolgie kann der Benutzer die Umgebung des mobilen Roboters mittels einer oder mehrerer angebrachter On-board Kameras virtuell mitverfolgen. Eine Software am eigenen Computer ermöglicht die Steuerung des Roboters aus der Ferne. Der Benutzer fühlt sich aufgrund dieser Möglichkeiten an diesem Ort des Roboters präsent, obwohl sich der Benutzer an einem anderen Ort aufhalten kann \cite{sheridan1992}.
%%%%%%%%%%%%%%%%%%%%%%%%%%%%

%Nun ist eine Erweiterung der Kommunikationswege um die Augensteuerung ein wichtiger Schritt. Jedoch bleibt die Kommunikation häufig auf Monitorsystemen mit präsentierenden Tastaturtafeln oder Symbolfeldern begrenzt und ist häufig zusätzlich abhängig von der menschlichen Unterstützung. Damit diese Abhängigkeit, die zwangsläufig durch die Schwere der Erkrankung bedingt ist, reduziert wird und der Grad der Selbstbestimmung sowie die Lebensqualität erhöht werden können, werden mobile Roboter verwendet. 

%%%%%%%%%%%%%%%%%%%%%%%%%%%%
%Die vorliegende Arbeit möchte einen Beitrag leisten um eine Augmentation der körperlichen Möglichkeiten von motorisch beeinträchtigten Personen im Alltag mittels eines Prototyps durch gezielte Interpertation von Augenbewegungen ein mobiles Robotersystems zu Steuern und die eigene Umgebung erkunden kann.

%%%%%%%%%%%%%%%%%%%%%%%%%%%%
%\section{Überblick}
%\label{section:überblick}
\begin{comment}
Der Mensch ist ein soziales Lebewesen und auf die Interaktion mit der Umgebung maßgeblich angewiesen. Geht die Möglichkeit der verbalen oder nonverbalen Interaktion krankheitsbedingt verloren, stellt dies einen tiefen Einschnitt in das Leben der Betroffenen dar. Die Augen des Menschen stellen bei vielen Erkrankungen Die Fähigkeit Interaktion mit Computersystemen  Die Augen des Menschen sind aufgrund ihrer anatomischen Lage und der nervalen Innervation bei vielen Erkrankungen häufig w  . Dadurch eröffnen sich 
\textcolor{red}{
Die vorliegende Abschlussarbeit konzentriert sich auf die Machbarkeit und Nützlichkeit der implementierten Robotersteuerung. Jedoch gelten für die Gestaltung der Benutzeroberfläche grundsätzliche Regeln. Hierbei werden die acht goldenen Regeln für die Benutzung freundlicher Gestaltung interaktiver Schnittstellen von Ben Schneidermann aufgeführt. Ferner werden Norman\'s sieben Grundsätzen beschrieben.
}
\end{comment}
%%%%%%%%%%%%%%%%%%%%%%%%%%%%%%
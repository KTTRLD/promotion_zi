Der allgemeine technische Fortschritt führt seit Jahrzehnten zu immer neuen Innovationen und stetigen Weiterentwicklungen bereits bestehender technischer Systeme. In der Luft- und Raumfahrttechnik ermöglichen Robotersysteme von der Erde aus die Erforschung fremder Planeten und Himmelskörpern. Am 12. November 2014 landete mit Philae erstmals eine von Menschen erbaute Apparatur weich auf einem Kometen\footnote{\url{https://de.wikipedia.org/wiki/Philae_(Sonde)} (letzter Aufruf: 14.12.2016)} und lieferte bis dato nicht bekannte Einblicke. Aber nicht nur die Luft- und Raumfahrttechnik profitiert von Robotersystemen, auch in der Medizin zeichnen sich zahlreiche vielversprechende Anwendungsfelder ab. Eine besondere Personengruppe, die von einer Weiterentwicklung mobiler Robotersysteme profitieren kann, sind Menschen mit Sprach- und Bewegungseinschränkungen. Hierbei besteht durch mobile Robotersysteme ein enormes Potential, die Handlungs- und Interaktionsfähigkeiten mit der näheren oder entfernteren Umgebung zu erweitert und die Qualität des Lebens zu verbessern. 
%Mobile Robotersysteme haben ein enormes Potential, zu dieser Gruppe der unterstützenden Systeme zu gehören. Ähnlich wie die Spracherzeugung oder elektronisch betriebene Rollstühle, können diese Systeme einen Beitrag zur Steigerung der Lebensqualität von Menschen mit Sprach- und Bewegungseinschränkungen leisten.

Traditionell werden Befehle oder Anweisungen und damit die eigentliche Kommunikation, von einem Menschen zu einem Computer, mittels einer Computermaus, einer Tastatur oder eines Joystick vermittelt. Für Menschen mit körperlichen Einschränkungen, speziell bei motorischen Beeinträchtigungen, können diese Methoden in Bezug auf die Nutzung einige Probleme bereiten. Zusätzliche Methoden sind notwendig, um dieses Problem zu lösen und damit diesen Menschen einen Zugang zu technischen Anwendungen zu ermöglichen. In den letzten Jahren konnte, beispielsweise die Spracherkennung, als neue Methode immer besser integriert werden und so Menschen mit Sehstörungen einen effizienteren Zugang zu technischen Systemen ermöglichen.
%Gal Sont is a programmer who suffers from motor neurone disease.
Ein prominentes Beispiel stellt die Firma Apple mit ihrer Spracherkennung namens SIRI zur Verfügung, die auf eine verbale Kommunikation, wie von Menschen praktiziert, ausgerichtet ist. Aber nicht nur die Sprache und die o.g. Geräte ermöglichen die Kommunikation mit einem Computer. Ein weiteres Beispiel ist der an Amyothropher Lateralsklerose (ALS) erkrankte Prof. Stephen Hawking. Dieser benutzt zur Kommunikation ein von Intel entwickeltes Programm mit dem Namen Assistive Context-Aware Toolkit (ACAT) als Schnittstelle \footnote{\url{https://01.org/acat},(Stand: 14.12.2016)} 
\footnote{\url{http://www.hawking.org.uk/the-computer.html} (Stand: 14.12.2016)}. ACAT ermöglicht durch die Darstellung einer Tastatur auf einem Bildschirm, die Auswahl von Buchstaben, welche eine Wortvorhersage ermöglicht und eine darauf aufbauende Sprachausgabe erzeugt. Eine alternative Lösung zu diesem verwendeten Prinzip zeigt die Arbeit von Eidam et. al. \vgl \cite{Eidam2016}. Hierbei wurden mittels eines Softwareprototypen Alltagsgegenstände in einer statischen Bildschirmszene dargestellt und passende Interaktionen durch Augengesten, die von einem Eyetracker erfasst wurden, ausgewählt. Hierdurch konnte gezeigt werden, dass eine Interaktion mit Alltagsgegenständen und mittels eines Eyetrackers eine Erweiterung der Handlungsmöglichkeiten bieten kann. Diese vorliegende Arbeit erweitert diesen Ansatz um ein mobiles Robotersystem, welches durch Augensteuerung die Interaktionsmöglichkeiten einer Person mit Sprach- und Bewegungseinschränkungen verbessern soll. Dieser Ansatz führt zu folgender Zielsetzung für die vorliegende Arbeit. 

% Einen Roboter nur mit den Augen in einer Umgebung zu steuern, kann eine  Aufgabe sein. Eventuell ist diese Umgebung sogar unbekannt oder nicht vertraut. Besonders Menschen mit Bewegungseinschränkungen können von einer Strategie der Steuerung von mobilen Robotersystemen nur mit den Augen profitieren. Die vorliegende Arbeit bietet zwei mögliche Strategien \bzw Steuerformen an, die es ermöglichen sollen, einen mobilen Roboter mithilfe eines Eyetracking-Systems zu steuern.


%Der Transfer von Informationen zwischen zwei Personen erfolgt in aller Regel durch die verbale Kommunikation.  

Das Austauschen und das Teilen von Gefühlen, innersten Absichten und Gedanken mit anderen Personen erscheint für viele Menschen selbstverständlich. Doch für Personen mit Sprach- oder Bewegungseinschränkungen stellt die Kommunikation eine große Herausforderung dar. Eine Personengruppe, die im Sommer 2014 besondere mediale Aufmerksamkeit durch die sogenannte \enquote{Ice Bucket Challenge} (deutsch Eiskübelherausforderung) \footnote{\url{https://de.wikipedia.org/wiki/ALS_Ice_Bucket_Challenge}} erhalten hatte, sind Personen mit \acf{als}. Ziel dieser Spendenkampagne war es, die Forschung \bzgl der Ätiologie der Erkrankung und eventuellen Therapiemöglichkeiten voranzutreiben. \acs{als} ist eine schwere neuromuskuläre Erkrankung, die das Zentralnervensystem schädigt und in aller Regel zu einem vorzeitigen Tod führt. Alleine in Deutschland starben im Jahr 2015, 2044 Personen (je 100.000 Einwohner) an dieser Erkrankung, wobei die Zahlen seit 1998 ansteigend sind, \vgl \acl{abb} \ref{fig:stat}.

\begin{figure}[ht]
   \begin{minipage}[t]{\linewidth} 
      \centering 
     %\includegraphics[scale=1]{bilder/grundlagen/als.pdf}
     \includegraphics[width=1\textwidth]{bilder/grundlagen/als.pdf}
   \end{minipage}% 
   \caption{Sterbefälle 1998 bis 2015 für die spinale Muskelatrophie und verwandte Syndrome (ICD-10 Code: G12), Statistisches Bundesamt (Destatis), 2017. In www.destatis.de (Thematische Recherche: Zahlen \& Fakten - Gesellschaft \& Staat - Gesundheit - Todesursachen - Dokumentart: Tabelle). Abrufdatum: 19. Februar 2017 \protect\footnotemark }\label{fig:stat} 
\end{figure} 

Die Erkrankung führt früher oder später in ein Stadium des \acf{lis}. Dies ist ein Zustand, der mit erhaltener kognitiver Fähigkeit einhergeht, bei gleichzeitig totaler oder subtotaler Bewegungseinschränkung. \acs{als} ist jedoch nur eine von vielen Ursachen, die zu diesem Zustand führen kann. Personen mit Bewegungseinschränkungen bleiben oftmals nur die Augen als einziger Kommunikationskanal zur Außenwelt erhalten. Eine Kommunikation nur auf der Grundlage von Augenbewegungen zu ermöglichen, stellt jedoch eine große Herausforderung dar. Technische Systeme, wie Eyetracking-Systeme können Augengesten ohne invasive Methoden sichtbar und interpretierbar machen. Diese Systeme erleichtern dadurch die Kommunikation mit der unmittelbaren Umgebung. Ein Schritt hin zu mehr Selbständigkeit kann in einer Kombination eines Eyetraking-Systems mit einem mobilen Robotersystem liegen. 
Mobile Robotersysteme können als eine Art Erweiterung der eigenen körperlichen Einschränkungen fungieren. Hier gibt es in den letzten Jahren zahlreiche kommerzielle Systeme.  Diese Systeme werden auch als \acf{tps} bezeichnet. 

Aus einer Verbindung zwischen Eyetracking-System und \acs{tps} können neue Möglichkeiten für Person mit Sprach- und Bewegungseinschränkungen entstehen, ihre Wünsche und Pläne eigenständig umzusetzen.

\footnote{Quelle: Sterbefälle 1998 bis 2015 für die spinale Muskelatrophie und verwandte Syndrome (ICD-Code: G12 nach ICD-10), Statistisches Bundesamt (Destatis), 2017. In www.destatis.de (Thematische Recherche: Zahlen \& Fakten - Gesellschaft \& Staat - Gesundheit - Todesursachen - Dokumentart: Tabelle). Abrufdatum:19. Februar 2017 \url{http://www.doublerobotics.com/}(Stand: Dezember 2016)}.
\footnotetext{\url{https://www.destatis.de/DE/ZahlenFakten/GesellschaftStaat/Gesundheit/Todesursachen/Tabellen/GestorbeneAnzahl.html}(letzter Aufruf: 19. Februar 2017)}
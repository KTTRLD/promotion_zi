\documentclass[11pt,a4paper,onecolumn,twoside,ngerman]{book}

% Konfiguration laden
% header.tex

\usepackage[a4paper,left=3.5cm,right=2.5cm,bottom=3cm,top=3cm]{geometry}
\usepackage[ngerman,english]{babel}
\usepackage{graphicx}
\usepackage{color}
\usepackage{import}
\usepackage{amsmath,amssymb}

\usepackage[numbers]{natbib}% von Author hinzugefügt
%\usepackage{amsmath,amssymb,subfigure}
 
\usepackage{verbatim} % von Autor ergänzt

\usepackage{needspace}% für itemize seiten neu

%%%%%%%%%%%%%%%%%%%%%%%%%%%%%%%%%%%%%%%%%%%%%%%%%%%%%%%%
\usepackage{wasysym} % für das checked symbol Häkchen
% für Farben im allgemeinen
\usepackage[table]{xcolor}
\usepackage{tabularx}
\usepackage{colortbl}	

\usepackage{array,longtable}
\usepackage[alwaysadjust]{paralist}
\setdefaultitem{\textbullet}{\textbullet}{\textbullet}{\textbullet}

%%%%%%%%%%%%%%%%%%%%%%%%%%%%%%%%%%%%%%%%%%%%%%%%%%%%%%

% Theorem-Umgebungen
\usepackage[amsmath,thmmarks]{ntheorem}
%\usepackage{svg}
% Korrekte Darstellung der Umlaute
\usepackage[utf8]{inputenc}
\usepackage[T1]{fontenc}
\usepackage{microtype}% verbesserter Randausgleich
\usepackage[autostyle=true,german=quotes]{csquotes}
% Algorithmen
\usepackage[plain,chapter]{algorithm}
\usepackage{algorithmic}
\usepackage{enumerate}
%Abkürzungsvereichnis
\usepackage[printonlyused]{acronym}

%PDF anhängen
\usepackage{pdfpages}

% Bibtex deutsch
\usepackage{bibgerm}
%Crossrodes referenzes
\usepackage{xr}
% URLs
\usepackage{url}
%notes und grafiken
\usepackage[ngerman]{todonotes}
\setlength{\marginparwidth}{2cm}
%Liste im Text
\usepackage{paralist}
% Caption Packet
\usepackage[margin=0pt,font=small,labelfont=bf]{caption}
\usepackage[labelformat=simple]{subcaption}
\renewcommand\thesubfigure{(\alph{subfigure})} 
%\usepackage[margin=0pt,font=small,labelfont=bf]{subcaption}
% Gliederung einstellen
%\setcounter{secnumdepth}{5}
%\setcounter{tocdepth}{5}

\usepackage[explicit]{titlesec}
\usepackage{lmodern}
\usepackage{lipsum}
\definecolor{mygreen}{rgb}{0,0.6,0}
\definecolor{mygray}{rgb}{0.5,0.5,0.5}
\definecolor{mymauve}{rgb}{0.58,0,0.82}
\definecolor{pl_background}{rgb}{0.95,0.95,0.95}
\definecolor{pl_comment}{rgb}{0.12, 0.38, 0.18 }
\definecolor{pl_ifelse}{rgb}{0.74,0.74,.29}
\definecolor{pl_keyword}{rgb}{0.37, 0.08, 0.25}
\definecolor{pl_string}{rgb}{0.06, 0.10, 0.98}

%\usepackage{listings}

\usepackage{listings,xcolor}
% Vordefiniertes Programmlisting
\lstdefinestyle{myCustomJavaStyle}{
%\lstset{
language = java,
basicstyle = \small\sffamily,
backgroundcolor = \color{pl_background},
stringstyle = \color{pl_string},
keywordstyle = \color{pl_keyword}\bfseries,
commentstyle = \color{pl_comment}\itshape,
frame = lrbt,
numbers = left,
showstringspaces = false,
breaklines = true,
xleftmargin = 15pt
%emph = [1]{java},
%emphstyle = [1]\color{black},
%emph = [2]{if,and,or,else},
%emphstyle = [2]\color{pl_keyword}
}

\lstdefinestyle{myCustomMatlabStyle}{
  language=java,
  numbers=left,
  stepnumber=1,
  numbersep=8pt,
  tabsize=2,
  showspaces=false,
  showstringspaces=true,
  backgroundcolor=\color{white},
  basicstyle=\footnotesize,
  commentstyle=\itshape\color{blue!90!black},
  keywordstyle=\bfseries\color{red!40!black},
  %identifierstyle=\color{black},
  %stringstyle=\color{blue}
  %keywordstyle=\color{black}
}

%\usepackage{rotating}%rotation der Tabelle Autor
%\usepackage{pdflscape} %Autor


% Abstand swischen zwei Absätzen
%\parskip=0.5em
%\setlength{\parindent}{0pt}


\newlength\chapnumb
\setlength\chapnumb{2.5cm}

\titleformat{\chapter}[block]
{\normalfont\sffamily}{}{0pt}
{\parbox[b]{\chapnumb}{%
   \fontsize{80}{70}\selectfont\thechapter}%
  \parbox[b]{\dimexpr\textwidth-\chapnumb\relax}{%
    \raggedleft%
    \hfill{\LARGE#1}\\
    \rule{\dimexpr\textwidth-\chapnumb\relax}{0.4pt}}}
\titleformat{name=\chapter,numberless}[block]
{\normalfont\sffamily}{}{0pt}
{\parbox[b]{\dimexpr\textwidth\relax}{%
    \raggedleft%
    \hfill{\LARGE#1}\\
    \rule{\dimexpr\textwidth\relax}{0.4pt}}}

% Theorem-Optionen %
\theoremseparator{.}
\theoremstyle{change}
\newtheorem{theorem}{Theorem}[section]
\newtheorem{satz}[theorem]{Satz}
\newtheorem{lemma}[theorem]{Lemma}
\newtheorem{korollar}[theorem]{Korollar}
\newtheorem{proposition}[theorem]{Proposition}
% Ohne Numerierung
\theoremstyle{nonumberplain}
\renewtheorem{theorem*}{Theorem}
\renewtheorem{satz*}{Satz}
\renewtheorem{lemma*}{Lemma}
\renewtheorem{korollar*}{Korollar}
\renewtheorem{proposition*}{Proposition}
% Definitionen mit \upshape
\theorembodyfont{\upshape}
\theoremstyle{change}
\newtheorem{definition}[theorem]{Definition}
\theoremstyle{nonumberplain}
\renewtheorem{definition*}{Definition}
% Kursive Schrift
\theoremheaderfont{\itshape}
\newtheorem{notation}{Notation}
\newtheorem{konvention}{Konvention}
\newtheorem{bezeichnung}{Bezeichnung}
\theoremsymbol{\ensuremath{\Box}}
\newtheorem{beweis}{Beweis}
\theoremsymbol{}
\theoremstyle{change}
\theoremheaderfont{\bfseries}
\newtheorem{bemerkung}[theorem]{Bemerkung}
\newtheorem{beobachtung}[theorem]{Beobachtung}
\newtheorem{beispiel}[theorem]{Beispiel}
\newtheorem{problem}{Problem}
\theoremstyle{nonumberplain}
\renewtheorem{bemerkung*}{Bemerkung}
\renewtheorem{beispiel*}{Beispiel}
\renewtheorem{problem*}{Problem}

% Algorithmen anpassen %
\renewcommand{\algorithmicrequire}{\textit{Eingabe:}}
\renewcommand{\algorithmicensure}{\textit{Ausgabe:}}
\floatname{algorithm}{Algorithmus}
\renewcommand{\listalgorithmname}{Algorithmenverzeichnis}
\renewcommand{\algorithmiccomment}[1]{\color{grau}{// #1}}

% Zeilenabstand einstellen %
\renewcommand{\baselinestretch}{1.25}

% Floating-Umgebungen anpassen %
\renewcommand{\topfraction}{0.9}
\renewcommand{\bottomfraction}{0.8}

% Markierte Referenzen werden ausgeblendet
\usepackage[pdfborder={0 0 0}]{hyperref}
% Abkuerzungen richtig formatieren %
\usepackage{xspace}

% Leere Seite ohne Seitennummer, naechste Seite rechts
\newcommand{\blankpage}{ \clearpage{\pagestyle{empty}\cleardoublepage}}
%\newcommand{\blankpage}{ \clearpage{\pagestyle{empty}\clearpage}}

% Keine einzelnen Zeilen beim Anfang eines Abschnitts (Schusterjungen)
\clubpenalty = 10000

% Keine einzelnen Zeilen am Ende eines Abschnitts (Hurenkinder)
\widowpenalty = 10000 
\displaywidowpenalty = 10000

% EOF

% Shortcuts laden

\newcommand{\RM}[1]{\MakeUppercase{\romannumeral #1{.}}}
\newcommand{\vgl}{vgl.\@\xspace} 
\newcommand{\zB}{z.\nolinebreak[4]\hspace{0.125em}\nolinebreak[4]B.\@\xspace}
\newcommand{\bzw}{bzw.\@\xspace}
\newcommand{\dahe}{d.\nolinebreak[4]\hspace{0.125em}h.\nolinebreak[4]\@\xspace}
\newcommand{\etc}{etc.\@\xspace}
\newcommand{\evtl}{evtl.\@\xspace}
\newcommand{\ggf}{ggf.\@\xspace}
\newcommand{\bzgl}{bzgl.\@\xspace}
\newcommand{\so}{s.\nolinebreak[4]\hspace{0.125em}\nolinebreak[4]o.\@\xspace}
\newcommand{\iA}{i.\nolinebreak[4]\hspace{0.125em}\nolinebreak[4]A.\@\xspace}
\newcommand{\sa}{s.\nolinebreak[4]\hspace{0.125em}\nolinebreak[4]a.\@\xspace}
\newcommand{\su}{s.\nolinebreak[4]\hspace{0.125em}\nolinebreak[4]u.\@\xspace}
\newcommand{\ua}{u.\nolinebreak[4]\hspace{0.125em}\nolinebreak[4]a.\@\xspace}
\newcommand{\og}{o.\nolinebreak[4]\hspace{0.125em}\nolinebreak[4]g.\@\xspace}
\newcommand{\oBdA}{o.\nolinebreak[4]\hspace{0.125em}\nolinebreak[4]B.\nolinebreak[4]\hspace{0.125em}d.\nolinebreak[4]\hspace{0.125em}A.\@\xspace}
\newcommand{\OBdA}{O.\nolinebreak[4]\hspace{0.125em}\nolinebreak[4]B.\nolinebreak[4]\hspace{0.125em}d.\nolinebreak[4]\hspace{0.125em}A.\@\xspace}

\newcommand{\rl}{reinforcement learning\@\xspace}
\newcommand{\Rl}{Reinforcement learning\@\xspace}
\newcommand{\et}
{Eyetracking-System\@\xspace}
\newcommand{\etp}
{Eyetracking-Systeme\@\xspace}
\newcommand{\tp}
{Telepräsenzrobotersystem\@\xspace}
\newcommand{\tpp}
{Telepräsenzrobotersysteme\@\xspace}
\newcommand{\iV}
{iView X\textsuperscript{TM}\@\xspace}
\newcommand{\spb}{stimuluspräsentierend Bildschirm\@\xspace}
\newcommand\mathplus{+}

\begin{document}

\selectlanguage{ngerman}
\section*{01.11.2016 - Aktueller Bearbeitungsstand - Entwicklung einer Steuerung für mobile Roboter mithilfe eines Eyetrackers.}

Laut des Zeitplans zur Abschlussarbeit vom 04.09.2016, sollten zum aktuellen Zeitpunkt (nach Beendigung der 43. KW) die Kapitel Einführung, Grundlagen, Fragestellung und Testmerkmale abgeschlossen sein. Ferner sollte der Evaluationsbogen mit den Testmerkmalen fertigstellt sein. 
Hierzu ist festzustellen, dass die \enquote{komplette Fertigstellung}, wie im Zeitplan als Meilenstein formuliert, wohl zu stringent formuliert wurde. Es liegt aktuell zu den o.g. Kapiteln jeweils eine Entwurfsform vor, die jedoch noch ca. 5-7 Arbeitstage Überarbeitungzeit benötigt. Dahingehend ist der Zeitplan zu verändern. Geplant ist deshalb die komplette Fertigstellung der Kapitel auf die 1. KW 2017 zu verschieben (siehe überarbeiteter Zeitplan).
Aktuell läuft planmäßig die Fertigstellung der Steuerung, wobei die kontinuierliche Steuerung noch getestet werden muss. Die Präsenzphase kann wie geplant in der 46. KW stattfinden. 
Zur Klärung der Arbeitsfragen bezüglich Machbarkeit, Präzision, Handling der mobilen Robotersteuerung, sowie zur Klärung wie ermüdend diese Art der Steuerung ist und ob Unterschiede zwischen einer diskreten und kontinuierlichen Steuerungsform vorhanden sind, wurde ein Fragebogen mit insgesamt 18 Fragen konzipiert. 
Ferner wurde zur Quantifizierung einer Parcourbewältigungsaufgabe die zeitliche Dauer als Merkmal festgelegt. Hierzu wurde ein Testparcour besprochen und aufgebaut, welcher einmalig mit jeder Steuerungsart umfahren werden muss. Die dabei benötigte Zeit, ausgehend von einer festgelegten Start- und Stoppposition, sollte aufgezeichnet und verglichen werden. Dies wird aktuell noch implementiert.

Zusammenfassend ergibt sich für die nächsten 4 Wochen voraussichtlich keine Änderung der Zeitplanung. Zur ausstehenden Fertigstellung der o.g. Kapitel ist KW 1. der \enquote{Pufferphase} vorgesehen.


\begin{table}[htp]
\begin{longtable}{|p{1cm}|p{1.5cm}|p{1.5cm}|p{9cm}|p{0.5cm}|}
%\begin{tabular}{l|l|ll|c}
%\toprule
\multicolumn{5}{c}{\LARGE \bf{1. überarbeiteter Zeitplan - Abschlussarbeit
}}  \\
\hline
\rowcolor{orange} \bf Mlst. & \bf Von & \bf Bis & \bf Ziele & \\
%\midrule
%\cellcolor{green}
%\checked 

%\newline - Software-Testversion - Stabilität ausreichend zur Evaluation der Basis-Steuerung + Codeoptimierung + Testfälle besprechen. 

%\newline - Konzept der Arbeit fertigstellen und Unklarheiten besprechen.

\hline
1 &\multicolumn{2}{|c|}{\bf Bis 01.10.2016} &  - Anmeldung der Bachelorarbeit.
\newline - Evaluationsbogen mit Evaluationsinhalten festlegen. 
& \checked \newline \checked \\
\hline
2 & \multicolumn{2}{|c|}{KW 40} & - Entwurf des Kapitels Einführung.& \checked \\ 
\hline
3 & KW 41 & KW 43 & - Entwurf des Kapitels Fragestellung und Testmerkmale.
\newline - Fertigstellung Evaluationsbogen. & \newline \checked \newline \checked \\
\hline
4 & KW 44 & KW 45 & - Softwareimplementierung der Augensteuerung abschließen. & \\
\hline
5 & \multicolumn{2}{|c|}{KW 46} & - Präsenzphase in Hagen mit Testung und Evaluation der Augensteuerung. & \\
\hline
6 & KW 47 & KW 48 & - Fertigstellung Kapitel Implementierung.\newline - Evaluationsergebnisse zusammenfassen. & \\
\hline
7 & KW 49 & KW 52 & - Fertigstellung der letzten Kapitel. & \\
\hline
8 & KW 1 & KW 2 & - Fertigstellung der Kapitel Einführung, Grundlagen, Fragestellung und Testmerkmale.&\\
\hline
 & KW 3 & KW 4 & - Pufferphase. \newline - Abschluss der ersten Version der Abschlussarbeit und Abgabe zum ersten Korrekturlesen. &\\
\hline
9 & KW 5 & KW 8 & - Bearbeitung der Korrekturvorschläge mit Fertigstellung der zweiten Version. 
\newline - ggf. Abgabe zum zweiten Korrekturlesen.& \\
\hline
10 & KW 9 & KW 11 & - Bearbeitung der Korrekturvorschläge mit Fertigstellung der endgültigen Version. \newline - ggf. Abgabe zum letzten Korrekturlesen. &\\
\hline
11 &\multicolumn{2}{|c|}{\bf Bis 03.04.2017} &  - Abgabe der Bachelorarbeit beim Prüfungsamt.& \\
\hline
\end{longtable}
\end{table}
\end{document}

